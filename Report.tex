\documentclass[12pt,a4paper]{scrreprt}

%\usepackage{fontspec}
%\usepackage{eurosym}
%\usepackage{amssymb}
%\usepackage{eurosym}
%\usepackage{geometry}
%\usepackage{mathtools}
%\usepackage[english]{babel}
%\usepackage{microtype}
%\usepackage{polyglossia}
%\usepackage{longtable,booktabs}

%\usepackage{grffile}
%\usepackage{titlesec}

\usepackage{graphicx}
\usepackage{capt-of}

%\setlength{\leftskip}{25pt}
%\makeatother
% Make links footnotes instead of hotlinks
%\renewcommand{\href}[2]{#2\footnote{\url{#1}}}

% No paragraph indentation
% and set space between paragraphs
\setlength{\parindent}{0pt}
\setlength{\parskip}{1em plus 2pt minus 1pt}
\setlength{\emergencystretch}{3em} % Prevent overfull lines
% Chapter: 0, section: 1, subsection: 2 etc
\setcounter{secnumdepth}{2}
\setcounter{tocdepth}{2}

\begin{document}
\pagenumbering{roman}
\begin{titlepage}
  \begin{center}

    \includegraphics[width=40mm]{tu}\par\vspace{0.8cm}
    \textsc{\LARGE Tribhuwan University}\\[1cm]
    \textsc{\bfseries INSTITUTE OF ENGINEERING}\\
    \textsc{\bfseries PULCHOWK CAMPUS}\\
    \textsc{Pulchowk, Lalitpur}\\
    \vspace{0.8cm}

    \textsc{\Large A Case Study}\\[0.5cm]
    \textsc{On}\\
    {\huge \bfseries Paila Technology\\[0.4cm]}

    \vfill
    {\Large \bfseries Submitted By:\\}
    \noindent\makebox[\textwidth]{%
      \begin{tabular}{lr}%
        Aashutosh Poudel & \texttt{072BCT502}\\
        Dinesh Bhattarai & \texttt{072BCT512}\\
        Krishna Upadhyay & \texttt{072BCT517}\\
        Rupesh Shrestha  & \texttt{072BCT530}\\
        Simon Dahal  & \texttt{072BCT538}\\
        Yogesh Rai  & \texttt{072BCT548}\\
    \end{tabular}}\\[1cm]

    {\Large \bfseries Submitted To:\\}
    \textsc{Department of Mechanical Engineering} \\
    \textsc{Lalitpur, Nepal}
    \\[1cm]

    \vfill

    {\large \today}

  \end{center}
\end{titlepage}

\chapter*{Acknowledgements}
\thispagestyle{empty}
We would like to express our deepest gratitude to Department of Electronics and Computer Engineering, Pulchowk Campus for providing us this opportunity of collaborative undertaking which has helped us gain valuable understanding of the courses we have studied. The resources, guidance and help from department has allowed us to gain both experience and skills.

We are highly indebted to our supervisors Mrs. Bibha Sthapit and Mr. Anil Verma for their continuous guidance and suggestions.

Finally, we would like to thank our friends and families for their support and encouragement that will lead to completion of this project.

- Team Members

\clearpage
\tableofcontents
\listoffigures
\listoftables
\clearpage
\pagenumbering{arabic}

\chapter{Introduction}
"Paila Technology", first of its kind in Nepali market, is a robotics and AI startup established on December 12, 2016 by Bijay Raut along with five other engineering graduates from IOE, Pulchowk Campus. The company started with a core team of seven members and the team has grown ever since. The company aims to produce world class robotics and industrial automation products that are suitable for businesses.
The company started with Automatic Dhara, automation for homes and have now transitioned into robotics by presenting their flagship project Pari, a robot for Nepal SBI Digital InTouch branch at Durbar Marg. Pari is a product that created a new market segment of robotics in Nepali market.

Paila is an entrepreneurial venture started with the prime motive to meet the demands of the market by developing a viable business model around some innovative products, services processes and platforms integrated with Artificial Intelligence and Robotics. The business model was a huge success in context of Nepal and the customerbase of the company is on rise. In a short span of time, the company has delivered products and services to some of the biggest names in the country including SBI Nepal Limited, nLocate, Gyani Traders and more to count.

\section{Vision}
The primary vision of "Paaila" can be considered as production of human friendly robots and to aid in the technological development of the country by embracing AI in different fields. It wants to aid other companies in integrating AI in their products as well to make AI and robotics available to everyone.

\section{Objectives}
The major objectives set by the company are:
\begin{enumerate}
  \item Be the best workplace for robotics and AI
  \item Be the best option for businesses seeking robotics Services
  \item Help Nepali companies integrate AI into their products and services.
\end{enumerate}

\chapter{Products and Services}
\section{Variable Frequency Drive}
\begin{enumerate}
  \item Introduction \\
    A Variable Frequency Drive (VFD) is a type of motor controller that drives an electric motor by varying the frequency and voltage supplied to the electric motor. Other names for a VFD are variable speed drive, adjustable speed drive, adjustable frequency drive, AC drive, and inverter. As the application's motor speed requirement change, the VFD can simply turn up or down the motor speed to meet the speed requirement.

  \item Working\\
    The first stage of a Variable Frequency AC Drive, or VFD, is the Converter. The converter is a comprised of six diodes, which are similar to check values used in plumbing systems. They allow current to flow in only one direction; the direction shown by the arrow in the diode symbol. For example, whenever A-phase voltage (voltage is similar to pressure in plumbing systems) is more positive than B or C phase voltages, then that diode will open and allow current to flow. When B-phase becomes more positive than A-phase, then the B-phase diode will open and the A-phase diode will close. The same is true for the 3 diodes on the negative side of the bus. Thus, we get six current "pulses" as each diode opens and closes. This is called a "six-pulse VFD", which is the standard configuration for current Variable Frequency Drives.

    \begin{center}
        \includegraphics{vfd1}
        \captionof{figure}{Converter in a VFD}
    \end{center}

    If the drive is operating on a 480V power system, then the 480V rating is "rms" or root-mean-squared. The peaks on a 480V system are 679V. The VFD dc bus has a dc voltage along with an AC ripple. The voltage runs between approximately 580V and 680V.

    \begin{center}
        \includegraphics{vfd2}
        \captionof{figure}{Converter with DC Bus in a VFD}
    \end{center}

    A capcitor is added to get rid of the AC ripple on the DC bus. A  capacitor operates in a similar fashion to a reservoir or accumulator in a plumbing system. This capacitor absorbs the ac ripple and delivers a smooth dc voltage. The AC ripple on the DC bus is typically less than 3 Volts. Thus, the voltage on the DC bus becomes "approximately" 650VDC. The actual voltage will depend on the voltage level of the AC line feeding the drive, the level of voltage unbalance on the power system, the motor load, the impedance of the power system, and any reactors or harmonic filters on the drive. \\

    The diode bridge converter that converts AC-to-DC, is sometimes just referred to as a converter. The converter that converts the dc back to ac is also a converter, but to distinguish it from the diode converter, it is usually referred to as an "inverter". It is common in the industry to refer to any DC-to-AC converter as an inverter.

    \begin{center}
        \includegraphics{vfd3}
        \captionof{figure}{Converter, DC Bus, and inverter in a VFD}
    \end{center}

    When one of the top switches in the inverter is closed, that phase of the motor is connected to the positive dc bus and the voltage on that phase becomes positive. When one of the bottom switches in the converter is closed, that phase is connected to the negative dc bus and becomes negative. Thus, any phase on the motor can be made positive or negative at will and can thus generate any frequency. Also, any phase can be made positive, negative, or zero.

    \begin{center}
      \includegraphics{vfd4}
      \captionof{figure}{Output response of a VFD}
    \end{center}

    The output from the VFD is of "rectangular" wave form. VFD's do not produce a sinusoidal output. This rectangular waveform would not be a good choice for a general purpose distribution system, but is perfectly adequate for a motor.

  \item Benefits
    \begin{itemize}
      \item Electric motor systems are responsible for more than 65\% of the power consumption in industry. Optimizing motor control systems using VFDs can reduce energy consumption and energy costs.
      \item Using VFDs, equipments can be operated at optimimum and efficient speeds, thereby extending the equipment life and reducing maintenance costs.
    \end{itemize}

  \item VFDs by Paila Technology
    \begin{itemize}
      \item Current Market in Nepal
        \begin{itemize}
          \item VFDs for \textit{Allo} thread producing machine
          \item VFDs for changing speed of DC motors in tempos
          \item VFDs for automatic brick producing machines
        \end{itemize}

      \item Features
        \begin{itemize}
          \item Customizable for different applications
          \item Production against possible drive failures
          \item Quality support and maintenance
          \item Compatible with single phase supply
          \item Made for Nepali Industries
        \end{itemize}

      \item Technical Specification
        \begin{center}
          \begin{tabular}{|c|c|}
            \hline
            Parameters & Values \\
            \hline
            Input Voltage & 380 to 480V 3ph or 230V single phase \\
            Input Frequency & 50Hz or 60Hz \\
            Output Voltage & 0 to rated line voltage \\
            Output Frequency & 0.5Hz to 200Hz \\
            Switching Frequency & 3kHz to 12kHz \\
            Rated Power & 0.5HP to 15 HP \\
            Control Method & Linear V/f \\
            Display & 7-segment or Alphanumeric LCD \\
            Digital Inputs & 5V DC optically isolated \\
            Analog Inputs & 0 to 5V \\
            \hline
          \end{tabular}
        \end{center}
    \end{itemize}
    Resources: https://www.vfds.com/blog/what-is-a-vfd
\end{enumerate}
\chapter{Organization Structure}

An organizational structure is a system that outlines how certain activities are directed in order to achieve the goals of an organization. These activities can include rules, roles and responsibilities. The organizational structure also determines how information flows from level to level within the company. For example, in a centralized structure, decisions flow from the top down, while in a decentralized structure, the decisions are made at various levels. \\

Organizational structure defines a specific hierarchy within an organization, and businesses of all shapes and sizes use it heavily. A successful organizational structure defines each employee's job and how it fits within the overall system. This structuring provides a company with a visual representation of how it is shaped and how it can best move forward in achieving its goals. \\

At its highest level, an organizational structure is either centralized or decentralized. Traditionally, organizations have been structured with centralized leadership and a defined chain of command. The military, for example, is an organization famous for its highly centralized structure, with a long and specific hierarchy of superiors and subordinates. However, there has been a rise in decentralized organizations, as is the case with many technology startups. This allows the companies to remain fast, agile and adaptable, with almost every employee receiving a high level of personal agency. \\

Paaila Technology is a fast paced company with core focus on AI, robotics, and industrial automation. In a short span of time, it has delivered products and services to some of the biggest brands in the country – SBI Nepal Limited, nLocate, Gyani Traders and many more. Paaila's popular flagship product 'Pari', a humanoid robot was deployed in providing customer service at SBI Bank in 2017. At its early stage, Paaila has captured the imagination of local Nepali market and acquired notable media attention by providing customer service through a humanoid robot. \\

\begin{center}
    \includegraphics[scale=0.45]{paila-group}
    \captionof{figure}{The founding members of Paila Technology src:startupsnepal.com}
\end{center}

Being an entrepreneurial venture started with a newly emerged concept in the marketplace, the prime motive of the advent of Paaila was to meet the demands and needs of the market by developing innovative products, services, processes and platforms relating to Artificial Intelligence and Robotics in context of Nepal. Being a startup, the company was initially formed with a small team of 7 members with a common interest. The company has a team of co-founders to secure key-skills, financial
resources, and other elements to conduct research on the target market while there are technical experts from different fields of engineering working together to think and develop the new ideas and innovative product concepts. In addition, the company consists of a management team which is responsible for major decisions on company’s plans, organization, staffing, coordination and control. \\

\begin{enumerate}
    \item Management Model
    A hybrid model of management comprising of a mixture of two management models, hierarchical and team efforts management model is followed. The overall structure can be taken as a hierarchy with Chairperson on the top, followed by the Chief Executive Officer(CEO), co-founders, other chief officers, engineers and finally the interns, staffs and other contract-based workers. But, to ensure better execution of projects, the organization follows a team efforts management model where the organization comprises of different teams based on the tasks. Individual groups give their best to the projects, while the head monitors the performance of all groups.

    \begin{center}
        \includegraphics[scale=0.28]{team-based-MM}
        \captionof{figure}{Team based Management Model}
    \end{center}

\end{enumerate}


\chapter{Personnel Management}
\section{Recruitment, Selection and Training}
\section{HR Divisions}
\subsection{Mechanical Division}
\subsection{Embedded Division}
\subsection{Language Processing}
\subsection{Operation Management}

\section{Motivation}

\chapter{Finance Management}
\chapter{Software Engineering Methodology}

\chapter{Organizational Challenges}
\section{Lack of expertise and difficulty to retain them}
\section{Lack of raw materials}
\section{Insufficient training data-sets}

\chapter{Recommendation}
\section{Outsourcing}
\section{Extending HR base for better responsibility division}

\chapter{Conclusion}
\chapter*{References}
\bibliography{bibliography}
\end{document}
