\documentclass[12pt,a4paper]{scrreprt}

%\usepackage{fontspec}
%\usepackage{eurosym}
%\usepackage{amssymb}
%\usepackage{eurosym}
%\usepackage{geometry}
%\usepackage{mathtools}
%\usepackage[english]{babel}
%\usepackage{microtype}
%\usepackage{polyglossia}
%\usepackage{longtable,booktabs}

%\usepackage{grffile}
%\usepackage{titlesec}

\usepackage{graphicx}

%\setlength{\leftskip}{25pt}
%\makeatother
% Make links footnotes instead of hotlinks
%\renewcommand{\href}[2]{#2\footnote{\url{#1}}}

% No paragraph indentation
% and set space between paragraphs
\setlength{\parindent}{0pt}
\setlength{\parskip}{1em plus 2pt minus 1pt}
\setlength{\emergencystretch}{3em} % Prevent overfull lines
% Chapter: 0, section: 1, subsection: 2 etc
\setcounter{secnumdepth}{2}
\setcounter{tocdepth}{2}

\begin{document}
\pagenumbering{roman}
\begin{titlepage}
  \begin{center}

    \includegraphics[width=40mm]{tu}\par\vspace{0.8cm}
    \textsc{\LARGE Tribhuwan University}\\[1cm]
    \textsc{\bfseries INSTITUTE OF ENGINEERING}\\
    \textsc{\bfseries PULCHOWK CAMPUS}\\
    \textsc{Pulchowk, Lalitpur}\\
    \vspace{0.8cm}

    \textsc{\Large A Case Study}\\[0.5cm]
    \textsc{On}\\
    {\huge \bfseries Paila Technology\\[0.4cm]}

    \vfill
    {\Large \bfseries Submitted By:\\}
    \noindent\makebox[\textwidth]{%
      \begin{tabular}{lr}%
        Aashutosh Poudel & \texttt{072BCT502}\\
        Dinesh Bhattarai & \texttt{072BCT512}\\
        Krishna Upadhyay & \texttt{072BCT517}\\
        Rupesh Shrestha  & \texttt{072BCT530}\\
        Simon Dahal  & \texttt{072BCT538}\\
        Yogesh Rai  & \texttt{072BCT548}\\
    \end{tabular}}\\[1cm]

    {\Large \bfseries Submitted To:\\}
    \textsc{Department of Mechanical Engineering} \\
    \textsc{Lalitpur, Nepal}
    \\[1cm]

    \vfill

    {\large \today}

  \end{center}
\end{titlepage}

\chapter*{Acknowledgements}
\thispagestyle{empty}
We would like to express our deepest gratitude to Department of Electronics and Computer Engineering, Pulchowk Campus for providing us this opportunity of collaborative undertaking which has helped us gain valuable understanding of the courses we have studied. The resources, guidance and help from department has allowed us to gain both experience and skills.

We are highly indebted to our supervisors Mrs. Bibha Sthapit and Mr. Anil Verma for their continuous guidance and suggestions.

Finally, we would like to thank our friends and families for their support and encouragement that will lead to completion of this project.

- Team Members

\clearpage
\tableofcontents
\listoffigures
\listoftables
\clearpage
\pagenumbering{arabic}

\chapter{Introduction}
"Paila Technology", first of its kind in Nepali market, is a robotics and AI startup established on December 12, 2016 by Bijay Raut along with five other engineering graduates from IOE, Pulchowk Campus. The company started with a core team of seven members and the team has grown ever since. The company aims to produce world class robotics and industrial automation products that are suitable for businesses.
The company started with Automatic Dhara, automation for homes and have now transitioned into robotics by presenting their flagship project Pari, a robot for Nepal SBI Digital InTouch branch at Durbar Marg. Pari is a product that created a new market segment of robotics in Nepali market.

Paila is an entrepreneurial venture started with the prime motive to meet the demands of the market by developing a viable business model around some innovative products, services processes and platforms integrated with Artificial Intelligence and Robotics. The business model was a huge success in context of Nepal and the customerbase of the company is on rise. In a short span of time, the company has delivered products and services to some of the biggest names in the country including SBI Nepal Limited, nLocate, Gyani Traders and more to count.

\section{Vision}
The primary vision of "Paaila" can be considered as production of human friendly robots and to aid in the technological development of the country by embracing AI in different fields. It wants to aid other companies in integrating AI in their products as well to make AI and robotics available to everyone.

\section{Objectives}
The major objectives set by the company are:
\begin{enumerate}
  \item Be the best workplace for robotics and AI
  \item Be the best option for businesses seeking robotics Services
  \item Help Nepali companies integrate AI into their products and services.
\end{enumerate}

\chapter{Products and Services}


\chapter{Organization Structure}
\chapter{Personnel Management}
\section{Recruitment, Selection and Training}
\section{HR Divisions}
\subsection{Mechanical Division}
\subsection{Embedded Division}
\subsection{Language Processing}
\subsection{Operation Management}

\section{Motivation}

\chapter{Finance Management}
\chapter{Software Engineering Methodology}

\chapter{Organizational Challenges}
\section{Lack of expertise and difficulty to retain them}
\section{Lack of raw materials}
\section{Insufficient training data-sets}

\chapter{Recommendation}
\section{Outsourcing}
\section{Extending HR base for better responsibility division}

\chapter{Conclusion}
\chapter*{References}
\bibliography{bibliography}
\end{document}
